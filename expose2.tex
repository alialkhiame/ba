\documentclass[a4paper,12pt]{report}

% Pakete
\usepackage[utf8]{inputenc} % Für UTF-8-Kodierung
\usepackage[T1]{fontenc}    % Für Schriftartkodierung
\usepackage[ngerman]{babel} % Für deutsche Sprache
\usepackage{geometry}       % Für Seitenränder
\usepackage{setspace}       % Für Zeilenabstand
\geometry{a4paper, left=30mm, right=30mm, top=30mm, bottom=30mm}

\title{Exposé zur Bachelorarbeit}
\author{Alkhiami Ali}
\date{\today}

\begin{document}

\maketitle

\section*{Arbeitstitel}
Entwicklung und Implementierung eines Modells zur Erfassung und Bewertung der User Experience am Beispiel der Anwendung eDok des LWV Hessen.

\section*{Problemstellung}
In der heutigen digitalen Welt ist die User Experience (UX) ein entscheidender Faktor für den Erfolg von Softwareanwendungen. Insbesondere in behördlichen Anwendungen wie eDok, das von der LWV Hessen verwendet wird, ist es essenziell, die UX zu erfassen und zu optimieren. Bisher existieren jedoch nur wenige spezialisierte Modelle zur systematischen Erfassung und Bewertung der UX in solchen Anwendungen.

\section*{Zielsetzung}
Ziel dieser Bachelorarbeit ist es, ein Modell zur Erfassung der UX in der Anwendung eDok zu entwickeln und zu implementieren. Dies beinhaltet die Auswahl und Anpassung geeigneter Methoden sowie die Anonymisierung und Auswertung der erhobenen Daten.

\section*{Methodik}
\begin{itemize}
    \item Literaturrecherche zu bestehenden UX-Modellen und Methoden.
    \item Entwicklung eines spezifischen Modells zur UX-Erfassung für eDok.
    \item Durchführung von Fragebögen, Interviews und Usability-Tests.
    \item Anonymisierung und Analyse der gesammelten Daten.
\end{itemize}

\section*{Zeitplan}
\begin{itemize}
    \item \textbf{Woche 1-2:} Themenfindung und Literaturrecherche.
    \item \textbf{Woche 3-4:} Konzeption und Methodik.
    \item \textbf{Woche 5-6:} Implementierung und Datenerhebung.
    \item \textbf{Woche 7:} Analyse der Ergebnisse.
    \item \textbf{Woche 8:} Abschluss und Vorbereitung des Schreibens.
    \item \textbf{Schreibphase (4 Wochen):} Einleitung, Methodik, Ergebnisse, Diskussion, Schlussfolgerung.
\end{itemize}

\section*{Literaturverzeichnis}
\begin{itemize}
    \item Nielsen, J., Norman, D. A. (2013). \textit{The Definition of User Experience (UX)}.
    \item Interaction Design Foundation (2020). \textit{User Experience (UX) – Einführung}.
    \item Usability.gov (2019). \textit{User Experience Basics}.
\end{itemize}

\end{document}
